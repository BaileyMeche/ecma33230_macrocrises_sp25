\documentclass[12pt]{article}
\usepackage{amsmath, graphicx, caption,array, amsthm}
\usepackage{amsfonts, xcolor, physics, listings,verbatim}
\usepackage{amssymb,empheq, mathrsfs, comment, subfig, hyperref, url, fancyhdr, tikz, booktabs, geometry, enumitem, textcomp, subfig}
\usepackage[T1]{fontenc} % for \symbol{92} 
% Command "alignedbox{}{}" for a box within an align environment
% Source: http://www.latex-community.org/forum/viewtopic.php?f=46&t=8144
\newlength\dlf  % Define a new measure, dlf
\newcommand\alignedbox[2]{
% Argument #1 = before & if there were no box (lhs)
% Argument #2 = after & if there were no box (rhs)
&  % Alignment sign of the line
{
\settowidth\dlf{$\displaystyle #1$}  
    % The width of \dlf is the width of the lhs, with a displaystyle font
\addtolength\dlf{\fboxsep+\fboxrule}  
    % Add to it the distance to the box, and the width of the line of the box
\hspace{-\dlf}  
    % Move everything dlf units to the left, so that & #1 #2 is aligned under #1 & #2
\boxed{#1 #2}
    % Put a box around lhs and rhs
}
}

\addtolength{\oddsidemargin}{-1in}
\addtolength{\evensidemargin}{-1in}
\addtolength{\textwidth}{1.75in}
\addtolength{\topmargin}{-1in}
\addtolength{\textheight}{1.75in}
\newcommand{\contra}{$\rightarrow\leftarrow$}
\newcommand{\tb}{  \textbackslash  }
\newcommand{\bj}{\ \Longleftrightarrow \ }

%bailey meche
\begin{document}
%bailey meche
	\begin{center}
		ECMA 33230: Macroeconomic Crises - Spring 2025\\
        Problem Set 3: Bank Runs \\
		Due Date: April 28, 2025 \\
        Bailey Meche
	\end{center}

\section*{1. (5 points) Banks as liquidity providers:}

\begin{enumerate}[label=(\alph*)]
    \item (2 points) Define ``liquidity'' in words.
    \subsubsection*{Solution}

    Liquidity measures how easily, quickly, and cheaply you can convert an asset into cash. 

    \item (3 points) Decide for each of the following assets if it’s relatively liquid or relatively illiquid: a factory plant, stocks, bonds, balance in your checking account, a piece of land, real estate, a gold ring, Bitcoin.
     \subsubsection*{Solution}

     
\end{enumerate}

\section*{2. (15 points)}

An asset that is purchased in period 0 pays out a return $R_1$ in period 1 or a return $R_2$ in period 2. Assume $R_2 \geq R_1$ and that $R_2$ is only earned if the asset is not liquidated in period 1. Imagine a consumer has preferences $(1 - \theta)u(c_1) + \theta u(c_2)$, where $\theta \in \{0, 1\}$ 
is a bivariate iid random variable with $\Pr(\theta = 0) = \pi = 0.25$. A consumer invests $x = 2$ in period 0 but has the choice between two assets. Asset A gives returns $R_{1}^A = 1$, $R_{2}^A = 2$ and asset B earns the returns $R_{1}^B = 1.25$, $R_{2}^B = 1.85$.

\begin{enumerate}[label=(\alph*)]
    \item (3 points) Derive formally the household’s expected utility function $EU$ in terms of $\pi$, $c_1$ and $c_2$, starting from the preferences given in the question. Show all steps of your derivation.

    \subsubsection*{Solution}
    \begin{align*}
         EU &= (1 - \theta)u(c_1) + \theta u(c_2)
    \end{align*}

    \item (2 points) What is the degree of liquidity of an asset in terms of $R_1$ and $R_2$? What is the degree of liquidity of cash? Is cash liquid or illiquid?
    \subsubsection*{Solution}

    \begin{align*}
        \text{degree of liquidity for A} &= \frac{R_1}{R_2} = \frac{1}{2} = 0.5
        \\ \text{degree of liquidity for B} &= \frac{R_1}{R_2} = \frac{1.25}{1.85} = 0.676
    \end{align*}
    Degree of liquidity of cash is 1. Cash is the baseline for liquidity as it is the standard liquid asset when considering US assets. 
    

    \item (1 point) Is asset A or asset B more liquid?
    \subsubsection*{Solution}

    From (b), we can see that asset $B$ is more liquid. 

    \item (2 points) What are the maximum consumption levels $c_1$ and $c_2$ if the entire investment is used to purchase asset A in period 0? What are $c_1$ and $c_2$ if asset B is purchased instead?
    \subsubsection*{Solution}

    If the entire investment $x=2$ is used to purchase asset A in period 0, the  maximum consumption levels are $c_1 = R_{1}^Ax = 2$ in $t=1$ and $c_2 = R_{2}^Ax = 4$ in $t=2$. 

    If the entire investment $x=2$ is used to purchase asset B in period 0, the  maximum consumption levels are $c_1 = R_{1}^Bx = 2.5$ in $t=1$ and $c_2 = R_{2}^Ax = 3.7$ in $t=2$. 


    \item (2 points) For $u(c) = 1 - \frac{1}{c}$, what is the expected utility for the consumer if the entire investment is used to purchase asset A? 
    What is the expected utility if the consumer invests everything in asset B instead? In which asset will the consumer optimally invest?
    \subsubsection*{Solution}

    If the entire investment $x=2$ is used to purchase asset A in period 0, then the expected utility is
    \begin{align*}
        EU_A &= (1 - \theta)u(c_1) + \theta u(c_2)
        \\ &= 0.75 \cdot u(2) + 0.25 \cdot u(4)
        \\ &= 0.75 \cdot  \left(1 - \frac{1}{2}\right) + 0.25 \cdot \left(1 - \frac{1}{4}\right)
        \\ &= 0.5625
    \end{align*}
    If the entire investment $x=2$ is used to purchase asset B in period 0, then the expected utility  is
    \begin{align*}
        EU_B &= (1 - \theta)u(c_1) + \theta u(c_2)
        \\ &= 0.75 \cdot u(2.5) + 0.25 \cdot u(3.7)
        \\ &= 0.75 \cdot  \left(1 - \frac{1}{2.5}\right) + 0.25 \cdot \left(1 - \frac{1}{3.7}\right)
        \\ &= 0.6324
    \end{align*}
    Given that $EU_B > EU_A$,  the risk-averse consumer will optimally invest in asset B.

    \item (5 points) Repeat part (e) for $u(c) = c - 2.5$. If you compare your results to part (e), why does this reflect that risk aversion creates demand for liquidity? Provide intuition.
    \subsubsection*{Solution}

     If the entire investment $x=2$ is used to purchase asset A in period 0, then the expected utility is
    \begin{align*}
        EU_A &= (1 - \theta)u(c_1) + \theta u(c_2)
        \\ &= 0.75 \cdot u(2) + 0.25 \cdot u(4)
        \\ &= 0.75 \cdot  \left(2 - 2.5\right) + 0.25 \cdot \left(4-2.5\right)
        \\ &= 0
    \end{align*}
    If the entire investment $x=2$ is used to purchase asset B in period 0, then the expected utility  is
    \begin{align*}
        EU_B &= (1 - \theta)u(c_1) + \theta u(c_2)
        \\ &= 0.75 \cdot u(2.5) + 0.25 \cdot u(3.7)
        \\ &= 0.75 \cdot  \left(2.5-2.5\right) + 0.25 \cdot \left(3.7-2.5\right)
        \\ &= 0.3
    \end{align*}
    Given that $EU_B > EU_A$,  the risk-averse consumer will optimally invest in asset B.
    
\end{enumerate}

\section*{3. (25 points)}

Consumers live for three periods, $t = 0, 1, 2$. In period 0, each consumer has an endowment of $B = 1.5$ and decides which part of the endowment to store $y$ or invest $x$. Investment earns a return $R = 1.3$ in period 2 and has a liquidation return of $L = 0.8$ in period 1. Consumers have utility $(1-\theta)u(c_1)+\theta u(c_2)$ where $u(c) = \frac{c^{1-\sigma}}{1-\sigma}$, with $\sigma = 3$, and come in two types: impatient consumers ($\theta = 0$) only care about period 1 consumption $c_1$, and patient consumers ($\theta = 1$) only care about period 2 consumption $c_2$. Types are revealed in period 1 and $\pi = \Pr(\theta = 0) = 0.5$ with $\theta \in \{0, 1\}$ iid.

\begin{enumerate}[label=(\alph*)]
    \item (4 points) Calculate the equilibrium levels for investment $x$, consumption $c_1$ and $c_2$ and the corresponding expected utility under autarky. How would these values change if liquidation was entirely ruled out?
    \subsubsection*{Solution}

    for consumer utility $(1-\theta)u(c_1)+\theta u(c_2)$ where $u(c) = \frac{c^{1-\sigma}}{1-\sigma}$, we have the household's problem in the autarky: 
    \begin{align*}
        &\max_{c_1, c_2, x,y} \pi u(c_1) + (1-\pi)u(c_2) && s.t. && \begin{cases}
            B  =x+y & t=0
            \\ c_1 \leq y + Lx = 1-x(1-L) & t=1
            \\ c_2 \leq y + Rx = 1+x(R-1) & t=2
        \end{cases}
        \\ &\max_{c_1, c_2, x,y} \frac{1}{2} \cdot  \frac{c_1^{1-\sigma}}{1-\sigma} + \frac{1}{2} \cdot  \frac{c_2^{1-\sigma}}{1-\sigma}
        \\ &-\frac{1}{4}\max_{c_1, c_2, x,y} \frac{1}{c_{1}^{2}}+\frac{1}{c_{2}^{2}}  && s.t. && \begin{cases}
             1.5 =x+y & t=0
            \\ c_1 \leq 1-0.2x  & t=1
            \\ c_2 \leq 1+0.3x & t=2
        \end{cases}
    \end{align*}
    Taking FOCs, we know that 
    \begin{align*}
         \frac{u'(c_1)}{u'(c_2)} &= \frac{(1-\pi)(R-1)}{\pi(1-L)} 
         \\  \frac{c_1^{-2}} {c_2^{-2}}&= 1.5
         \\ \frac{c_1}{c_2}&= \frac{1}{\sqrt{1.5}}\approx 0.816
         \\ c_2 &= \sqrt{1.5}c_1 
    \end{align*}
    This result lies in the interior solution $x\in(0,1)$, so using this to solve for $x$:
    \begin{align*}
       1+0.3x = c_2 &= \sqrt{1.5}c_1  = \sqrt{1.5}(1-0.2x)
       \\  1-\sqrt{1.5}&=x\left(-0.3-0.2\sqrt{1.5}\right)
       \\ x\ &=\ \frac{50\sqrt{1.5}-60}{3} \approx 0.4124
       \\ c_1 &= 1-0.2\frac{50\sqrt{1.5}-60}{3} = \frac{15-10\sqrt{1.5}}{3} \approx 0.9175
       \\ c_2 &= 1+0.3\frac{50\sqrt{1.5}-60}{3} = 5\sqrt{1.5}-5 \approx 1.1237
    \end{align*}

    If liquidation was ruled out, then impatient consumers $(\theta=0)$ are forced to move all consumption to period 2 since they cannot consume in $t=1.$ Such consumers can still choose to store or invest.
    \begin{align*}
        &-\frac{1}{4}\max_{c_1, c_2, x,y} \frac{1}{c_{1}^{2}}+\frac{1}{c_{2}^{2}}  && s.t. && \begin{cases}
             1.5 =x+y & t=0
            \\ c_1 \leq 1-x  & t=1
            \\ c_2 \leq 1+0.3x & t=2
        \end{cases}
    \end{align*}?????
    
    
    \item (2 points) Set up a consumer’s incomplete market problem in period 0 with the possibility to trade in period 1. What are the market equilibrium consumption levels $c_1$ and $c_2$ and the associated expected utility level?
\subsubsection*{Solution}



    \item (2 points) Compute the equilibrium levels for investment $x$, consumption $c_1$ and $c_2$ and the corresponding expected utility in an economy with financial intermediation. Assume the banking system achieves the good Nash equilibrium.
\subsubsection*{Solution}



    \item (8 points) Plot the feasible sets under autarky and under financial intermediation in a $(c_1, c_2)$-graph. Then add the indifference curves that are associated with your results from parts (a), (b) and (c) in the plot. Use Julia to make this plot and include your code.
\subsubsection*{Solution}



    \item (2 points) Why can maturity mismatch in bank’s balance sheet lead to a worse outcome than in part (c)? Provide intuition.
\subsubsection*{Solution}



    \item (3 points) Suppose the government wants to insure deposits by imposing a tax on early withdrawals to rule out bank runs. Explain how the government could do this. What is the tax rate the government sets to achieve the first best?
\subsubsection*{Solution}



    \item (4 points) Suppose there is deposit insurance in place. Calculate the consumption levels of a patient consumer who withdraws in period 1, in the two situations where more or less than the share of early consumers withdraw in period 1. Calculate the consumption levels of a patient consumer who withdraws in period 2, in the two situations where more or less than the share of early consumers withdraw in period 1.
    \subsubsection*{Solution}

    
\end{enumerate}

\section*{4. (14 points)}

A measure 1 of consumers live for three periods, $t = 0, 1, 2$. In period 0 each consumer has an endowment of \$3 and decides which part of the endowment to store $y$ or invest $x$. Investment earns a return $R = 1.1$ in period 2. Consumers have expected utility function $EU = \pi u(c_1)+(1-\pi)u(c_2)$. In period 1 it is revealed if a consumer is an early type and only cares about consumption $c_1$ or a late type and only values $c_2$, where $\pi$ is the probability of being an early consumer. Early consumers can sell their investments to late consumers at price $P$ in period 1.

\begin{enumerate}[label=(\alph*)]
    \item (2 points) Write down the period-by-period budget constraints for a consumer.
    \subsubsection*{Solution}

    \item (8 points) \textbf{Extra Credit:} Prove that $P = 1$ in equilibrium.
    \subsubsection*{Solution}

    
    \item (2 points) How much do early and late types consume in the market equilibrium?
    \subsubsection*{Solution}

    
    \item (2 points) Assume $\pi = 0.1$, $u(c) = \frac{c^{1-\gamma}}{1- \gamma}$, $\gamma = 2$. What is the expected utility in the market?
    \subsubsection*{Solution}


    
\end{enumerate}

\section*{5. (8 points) Bank Runs}

A demand deposit contract between bankers and consumers with endowment equal to 1 is $D = (c^B_1 = 1.04, c^B_2 = 1.06)$, where the subscripts refer to the period of withdrawal. The depositor Eva typically prefers to wait with a withdrawal until period 2, however she also knows that if a bank run occurs she can only withdraw 0.8 if she runs like everybody else or nothing if she is the only one who waits with the withdrawal to period 2.

\begin{enumerate}[label=(\alph*)]
    \item (4 points) Fill in the blank consumption level matrix to help Eva decide whether to run or not:
    \subsubsection*{Solution}

    
    \begin{center}
        \begin{tabular}{|c|c|c|}
            \hline
            & \textbf{Everybody else Runs} & \textbf{Everybody else Does Not Run} \\
            \hline
            \textbf{Eva Runs} & $(\ ,\ )$ & $(\ ,\ )$ \\
            \hline
            \textbf{Eva Does Not Run} & $(\ ,\ )$ & $(\ ,\ )$ \\
            \hline
        \end{tabular}
    \end{center}

    \item (2 points) How many (pure strategy) equilibria exist in this scenario? What are the associated payoffs?
    \subsubsection*{Solution}

    \item (2 points) Under what conditions do all consumers withdraw in period $t = 1$ in a good Nash equilibrium?
    \subsubsection*{Solution}

    
\end{enumerate}

\section*{6. (5 points)}

Why is the stockpiling and rationing of toilet paper at the beginning of the Corona Crisis sometimes referred to as a “run”? Explain the similarities to a bank run.
\subsubsection*{Solution}


\end{document}